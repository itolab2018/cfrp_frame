\chapter{はじめに}


\section{研究の背景}
近年は,マルチコプターは,農業,空撮,物資配達,救助といった様々な場面で用いられるようになってきた.遠隔操作が可能で,複数のプロペラや翼を搭載した機体である.現代において更なる技術革新と汎用性が求められている.


\section{研究の目的}
本研究室では,マルチコプター通称ドローンを設計から一貫し製作を行ってきた.これまでの研究目標としてきた校内を自律飛行案内するという目標を掲げ活動を行ってきた.中間目標としていた全日本飛行ロボットコンテストに出場し見事に優勝を飾ったが,いくつかの課題も残った.具体的に大会でのミッションとなっていた自動離着陸や,自動八の字飛行ができなかった.また,CFRPを用いたフレームにおいても大会規定であった機体重量に対し,自動飛行用のためのセンサーなどを搭載していた場合,重量規定を超えてしまっていた.そのためより軽く合成のあるフレーム製作が最も重要であり必要不可欠であると考えられた.平面積層のCFRPフレームに対し,より軽量で剛性のあるフレーム製作を目的とする.そのためこれまでの平面積層ではなく,立体構造のフレーム製作を行うこととした.


\begin{figure}[htbp]
  \begin{center}
    \includegraphics[width=120mm]{img/1.JPG}
    \end{center}
  \caption{カーボンフレームのマルチコプター}
 \label{fig:robot}
\end{figure}


\section{本論文の構成}
1章では,本研究の背景と簡略化した概要を示す.また,研究室での自立飛行マルチコプター製作においての自身の役割について述べる,全日本学生飛行ロボットコンテストについて述べる.3章では平面積層フレームの製作工程について述べる.4章では立体積層フレームについて述べる.5章では平面積層と立体積層フレームにおいての強度曲げ試験,および結果について述べる.6章ではフレーム製作についてのまとめを示す.